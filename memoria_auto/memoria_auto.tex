\documentclass[12pt,a4paper]{article}

%Opciones del texto
\usepackage[spanish]{babel}
\usepackage[utf8]{inputenc}
\setlength{\parindent}{3em} %Sangría
\setlength{\parskip}{0.4em} %Espaciado entre parrafos
\renewcommand{\baselinestretch}{1.1}
\usepackage[a4paper,total={6in, 9in}]{geometry}

%Encabezado, pie de página
\usepackage{fancyhdr}
\pagestyle{fancy}
\fancyhf{}
\lhead[\rightmark]{Automatización de Sistemas de Producción}
%\rhead[\rightmark]{SinteTiva} %encabezado a la derecha
\rfoot[]{\thepage} %Muestra la página


%Bibliografía
\bibliographystyle{unsrt}
\usepackage{hyperref}
\hypersetup{
    colorlinks=true,
    linkcolor=blue,
    filecolor=magenta,      
    urlcolor=cyan,
}
\urlstyle{same}


%Fotos, gráficos
\usepackage{tabularx, caption}
\usepackage{subcaption}
\usepackage{graphicx}
\graphicspath{ {images/} }

%Símbolos
\usepackage{eurosym} %Euro
\def\checkmark{\tikz\fill[scale=0.4](0,.35) -- (.25,0) -- (1,.7) -- (.25,.15) -- cycle;} 

%Añadir PDFs
\usepackage{pdfpages} 

%Comentarios
\usepackage{verbatim} 

%Código
\usepackage{listings} 

%Colores
\usepackage{colortbl}
\usepackage{array}
\usepackage{xcolor} 
\usepackage{color}
%Colores definidos
\definecolor{gray97}{gray}{.97}
\definecolor{gray75}{gray}{.75}
\definecolor{gray45}{gray}{.45}
\definecolor{miverde}{rgb}{0,0.6,0}
\definecolor{migris}{rgb}{0.5,0.5,0.5}
\definecolor{mimalva}{rgb}{0.58,0,0.82}

\lstset{ %
commentstyle=\color{miverde},    % Estilo de los comentarios
frame=single,	                   % Añade un marco al código
keywordstyle=\color{mimalva},       % estilo de las palabras clave
rulecolor=\color{black},         % Si no se activa, el color del marco puede cambiar en los saltos de línea entre textos que sea de otro color, por ejemplo, los comentarios, que están en verde en este ejemplo
backgroundcolor=\color{gray97},
breaklines=true,                 % Activa el salto de línea automático
%%stringstyle=\ttfamily,
%%basicstyle=\small\ttfamily,
%%showstringspaces = false,
%%basicstyle=\small\ttfamily,
}

%WBS -> Árbol
\usepackage{tikz}
\usetikzlibrary{arrows,shapes,positioning,shadows,trees}

\lstdefinestyle{customc}{
  belowcaptionskip=1\baselineskip,
  breaklines=true,
  frame=L,
  xleftmargin=\parindent,
  language=C,
  showstringspaces=false,
  basicstyle=\footnotesize\ttfamily,
  keywordstyle=\bfseries\color{green!40!black},
  commentstyle=\itshape\color{purple!40!black},
  identifierstyle=\color{blue},
  stringstyle=\color{orange},
}




%Para poder poner tildes en el código
\lstset{
     literate=%
         {á}{{\'a}}1
         {í}{{\'i}}1
         {é}{{\'e}}1
         {ý}{{\'y}}1
         {ú}{{\'u}}1
         {ó}{{\'o}}1
         {ě}{{\v{e}}}1
         {š}{{\v{s}}}1
         {č}{{\v{c}}}1
         {ř}{{\v{r}}}1
         {ž}{{\v{z}}}1
         {ď}{{\v{d}}}1
         {ť}{{\v{t}}}1
         {ň}{{\v{n}}}1                
         {ů}{{\r{u}}}1
         {Á}{{\'A}}1
         {Í}{{\'I}}1
         {É}{{\'E}}1
         {Ý}{{\'Y}}1
         {Ú}{{\'U}}1
         {Ó}{{\'O}}1
         {Ě}{{\v{E}}}1
         {Š}{{\v{S}}}1
         {Č}{{\v{C}}}1
         {Ř}{{\v{R}}}1
         {Ž}{{\v{Z}}}1
         {Ď}{{\v{D}}}1
         {Ť}{{\v{T}}}1
         {Ň}{{\v{N}}}1                
         {Ů}{{\r{U}}}1    
}

\begin{document}


%%%%%%%%%%%%%%%%%%%%%%% PORTADA
\begin{titlepage} 
%Estaría bien hacerla en word e incluir el PDF porque hacer portadas en LATEX es dificil.
%\includepdf{portada.pdf}
\end{titlepage}
%%%%%%%%%%%%%%%%%%%%%%%%%%%%%%%%%%%%%%%%%%%%%%%%%%%%%%%

\tableofcontents
\newpage

\lstset{language=C++} 
\section{SinteTiva}

\section{Manual de usuario}

\section{Desarrollo del proyecto}

\subsection{Funciones en la librería}

\subsubsection{detectaNota}

\subsubsection{pintaDeteccion}

\subsubsection{pintaTiempo}

\subsubsection{cmd slider}

\subsubsection{Selector}

%para poner una imagen
%\begin{figure}[h!]
%\centering
 % \includegraphics[width=8cm]{menu2.jpeg}
 % \caption{Menú 2. Composición de canciones.}
 % \label{fig:img1}
%\end{figure}

%para poner en negrita
%\textbf{Pantalla de inicio} 

%para hacer una numeracion
\begin{itemize} 
\item uno: aqui se habla de noseke
\item dos: aqui se habla de tal
\end{itemize}
%

%COMENTADO
\begin{comment}

\begin{figure}[h!]
\centering
  \includegraphics[width=10cm]{2_1.jpeg}
  \caption{Pantalla de inicio}
  \label{fig:img1}

  \centering
  \begin{subfigure}[b]{0.45\linewidth}
    \includegraphics[width=\linewidth]{2_2.jpeg}

  \end{subfigure}
  \begin{subfigure}[b]{0.45\linewidth}
    \includegraphics[width=\linewidth]{2_3.jpeg}

  \end{subfigure}
  \label{fig:units}


  \begin{subfigure}[b]{0.45\linewidth}
    \includegraphics[width=\linewidth]{2_4.jpeg}

  \end{subfigure}
  \begin{subfigure}[b]{0.45\linewidth}
    \includegraphics[width=\linewidth]{2_5.jpeg}


  \end{subfigure}
  \label{fig:units}

\end{figure}

\end{comment}

%\begin{lstlisting}[basicstyle=\footnotesize]

%\end{lstlisting}

%\begin{figure}[h!]
%  \includegraphics[width=8cm]{hola.jpeg}
%  \caption{Esquema}
%  \label{fig:img1}
%\end{figure}
%\begin{figure}[h!]
%  \includegraphics[width=8cm]{2.jpeg}
%  \caption{Esquema}
%  \label{fig:img1}
%\end{figure}

\end{document}
